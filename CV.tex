\documentclass[letterpaper,margin,line]{res}

\oddsidemargin -.5in
\evensidemargin -.5in
\textwidth=6.0in
\itemsep=0in
\parsep=0in

\newenvironment{list1}{
  \begin{list}{\ding{113}}{%
      \setlength{\itemsep}{0in}
      \setlength{\parsep}{0in} \setlength{\parskip}{0in}
      \setlength{\topsep}{0in} \setlength{\partopsep}{0in} 
      \setlength{\leftmargin}{0.17in}}}{\end{list}}
\newenvironment{list2}{
  \begin{list}{$\bullet$}{%
      \setlength{\itemsep}{0in}
      \setlength{\parsep}{0in} \setlength{\parskip}{0in}
      \setlength{\topsep}{0in} \setlength{\partopsep}{0in} 
      \setlength{\leftmargin}{0.2in}}}{\end{list}}

\usepackage[pdftex,colorlinks=true,linkcolor=black,urlcolor=blue]{hyperref}
\hypersetup{
pdftitle={Curriculum Vitae for Joshua Ryan Smith},
pdfauthor={Joshua Ryan Smith (jrsmith@cmu.edu)},
pdfcreator={pdfTeX}
}

\begin{document}

\name{Joshua Ryan Smith \vspace*{.1in}}

\begin{resume}
\section{\sc Contact Information}
\vspace{.05in}
\begin{tabular}{@{}p{2in}p{4in}}
5000 Forbes Ave 		& {\it Voice:} (919) 413.0396\\
Roberts Hall 232        	& {\it Email:} {jrsmith@cmu.edu} \\
Pittsburgh PA 15213-3890	& {\it Web:} \href{http://www.materials.cmu.edu}{www.materials.cmu.edu}
\end{tabular}


\section{\sc Education}

{\bf North Carolina State University}, Raleigh, North Carolina USA\\
\vspace*{-.1in}
\begin{list1}
\item[] Ph.D., Physics. December 2007. Dissertation ``Thermionic energy conversion and particle detection using diamond and diamond-like carbon surfaces.'' \\Co-Chairs: Griff Bilbro and Robert Nemanich
\item[] B.S., Physics,  May, 2002, \textit{Cum Laude}
\item[] B.S., Mathematics, May, 2002, \textit{Cum Laude}
\end{list1}

%{\bf North Carolina School of Science and Mathematics}, Raleigh, North Carolina USA\\
%\vspace*{-.1in}
%\begin{list1}
%\item[] Diploma, 1998
%\end{list1}


\section{\sc Skills} 
Ultrahigh vacuum (UHV) techniques and practices

\vspace*{-2.5mm}
Surface characterization: X-ray photoemission spectroscopy (XPS), Ultraviolet photoemission spectroscopy (UPS), Auger electron spectroscopy (AES)

\vspace*{-2.5mm}
Hydrogen passivation of silicon (100) using hot filament decomposition

\vspace*{-2.5mm}
Diamond-like carbon thin film growth using plasma enhanced CVD (ECR) technique

\vspace*{-2.5mm}
Hydrogen passivation of tetrahedral amorphous carbon (ta-C) using plasma enhanced CVD (ECR) technique

\vspace*{-2.5mm}
Software development: source control using git/subversion, test-driven development, project hosting with github: \href{http://github.com/jrsmith3}{http://github.com/jrsmith3}

\vspace*{-2.5mm}
Programming languages: Python, MATLAB

\vspace*{-2.5mm}
Linux system administration

\vspace*{-2.5mm}
Applications: AutoCAD, QCad, \LaTeX, OpenOffice.org, Comsol Multiphysics (formerly FEMLAB)

\vspace*{-2.5mm}
Languages: English (native), German (4 years high school and college)

\section{\sc Experience}
{\bf Carnegie Mellon University}, Pittsburgh, Pennsylvania, USA

\vspace{-.3cm}
{\em Postdoctoral Researcher} \hfill {\bf June 2008 - Present}\\
\begin{list1}
\item[] Project: Tip Directed, Field Assisted Nanomanufacturing (TFAN) technique based on scanning probe microscopy. Research focus: nanoscale milling of Si (100).
\item[] Managed our group's efforts to meet research metrics and secure Phase II funding. Includes Si sample prep and demonstrating nanolithography in H-passivation layer of the surface.
\item[] Developed strategy to meet Phase III research metrics: fabrication of single electron transistor using TFAN technique.
\item[] Built laboratory from scratch including project management of renovations, layout of equipment and systems, coordination of equipment delivery, and outfitting lab with tools and supplies.
\item[] Responsible for day-to-day operations of lab including management of three graduate students.
\item[] Designed, engineered, and built automated process gas delivery system for lab, and $H_{2}$ decomposition filament/gas doser assembly for UHV chambers.
\end{list1}


{\bf North Carolina State University}, Raleigh, North Carolina, USA

\vspace{-.3cm}
{\em Graduate Research Assistant} \hfill {\bf August 2002 - August 2007}\\
Ph.D. research and coursework.
\begin{list1}
\item[] Developed theory of electron transport within a vacuum thermionic energy converter with a negative electron affinity emitter. Implemented theory by writing MATLAB programs from scratch.
\item[] Completed Comsol Training, Advanced Modeling in FEMLAB in April 2005.
\item[] Coordinated operations at NCSU with Lockheed Martin for the IBEX mission (www.ibex.swri.edu). Received, processed, and shipped detector facets on time and within specification.
\item[] Developed hydrogen passivation procedure using plasma enhanced CVD (ECR) technique for ta-C facets for use in IBEX-Lo detector.
\item[] Coordinated AFM analysis of IBEX detector facets with the NCSU Analytical Instrumentation Facility.
\item[] Managed undergrad student for NCSU's REU program in summer 2005.
\end{list1}

{\em Undergraduate Researcher} \hfill {\bf August 2001 - May 2002}\\
Computational molecular dynamics using Monte Carlo method. Advisor: Dr. Lubos Mitas

{\em Research Experience for Undergraduates Program} \hfill {\bf May 2001 - August 2001}\\
Computational molecular dynamics using Monte Carlo method. Advisor: Dr. Lubos Mitas

{\em Tutor} \hfill {\bf August 1998 - May 2000}\\
Physics Tutorial Center, North Carolina State Tutorial Center

{\bf Triangle Learning Consultants}, Raleigh, North Carolina, USA

\vspace{-.3cm}
{\em Tutor} \hfill {\bf August 2000 - August 2002}\\
Tutored several high school students on an individual basis in the subjects of basic physics and mathematics from algebra to pre-calculus.

%{\bf Great Commission Ministries}, Columbus, Ohio, USA

%\vspace{-.3cm}
%{\em Counsler} \hfill {\bf June 2000 - July 2000}\\
%I was a counselor at a summer camp for high school students. I helped with some of the logistics for the camp.

{\bf Johnson Controls}, Charlotte, North Carolina, USA

\vspace{-.3cm}
{\em Engineering Assistant} \hfill {\bf Summer 1999, Summer 2000}\\
Drafted and modified technical documents and schematics using Visio and AutoCAD according to the engineers' specifications. Assisted in laying out the user interface for some of the projects' HVAC systems.


\section{\sc Honors and Awards} 
COMAP Mathematical Contest in Modeling 2002, Meritorious Submission

\vspace*{-2.5mm}
Eagle Scout Award, 1996


\section{\sc Professional Affiliations} 
Materials Research Society, Member.
American Physical Society, Member.

\section{\sc Extra\\ Curricular}
{\bf Graduate Physics Student Association (GPSA)}

\vspace{-.3cm}
{\em President} \hfill {\bf April 2005 - April 2006} \\
Led coordination of events to enrich the life of the physics graduate students. These events ranged from the department's holiday party to movie nights.

\vspace{-.3cm}
{\em University Graduate Student Assc. Representative} \hfill {\bf April 2003 - April 2005}\\
Represented the graduate physics students at the University Graduate Student Association meetings.

{\bf Hobbies}

\vspace{-.3cm}
Member of HackPittsburgh, a non-profit, community-based workshop that allows members to come together and share skills and tools to pursue creative projects. \href{http://www.hackpittsburgh.org}{www.hackpittsburgh.org}

\vspace{-.3cm}
Replaced clutch and pilot bearing in the transmission of my truck. April 2007

\vspace{-.3cm}
Spent a week with a professional welder learning MIG welding techniques. December 2003

%The UGSA representative is an elected position within the GPSA. During my time in the GPSA, I helped regenerate the defunct GPSA. I also helped acquire funds to purchase new computers for the first year graduate student offices. I also helped plan and execute social events for the graduate students as well as provide a forum for graduate physics students to present their work.

% {\bf University Graduate Student Association (UGSA)}
% 
% \vspace{-.3cm}
% {\em Political Action and Awareness Committee} \hfill {\bf August 2002 - Present}\\
% The UGSA is the organization that represents NC State graduate students. I am currently serving my second term as a member of the Political Action and Awareness committee. Last year this committee organized several service projects that NC State graduate students participated in. Also, I helped the committee gather information about how to lobby the North Carolina state legislature. This year the PAC is working on similar projects.


\section{\sc Publications}
Smith, J.R., Nemanich, R. Hydrogen termination and negative electron affinity of tetrahedral amorphous carbon. In preparation.

Smith, J.R., Bilbro, G., Nemanich, R. \href{http://dx.doi.org/10.1116/1.3125282}{Theory of space charge limited regime of thermionic energy converter with negative electron affinity emitter}. Journal of Vacuum Science and Technology B, 2009; 27: 1132-1141

Smith, J.R., Bilbro, G., Nemanich, R. \href{http://dx.doi.org/10.1103/PhysRevB.76.245327}{Considerations for a high performance thermionic energy conversion device based on an NEA emitter}. Physical Review B, 2007; 76: 245327-245332

Smith, J.R., Bilbro, G., Nemanich, R. \href{http://dx.doi.org/10.1016/j.diamond.2006.09.011}{Using negative electron affinity diamond emitters to mitigate space charge in vacuum thermionic energy conversion devices}. Diamond and Related Materials, 2006; 15: 2082-2085.

Smith, J.R., Bilbro, G., Nemanich, R. \href{http://dx.doi.org/10.1016/j.diamond.2005.12.057}{The effect of Schottky barrier lowering and nonplanar emitter geometry on the performance of a thermionic energy converter}. Diamond and Related Materials, 2006; 15: 870-874.

Smith, R.C., Seelecke, S., Ounaies, Z., Smith, J.R. \href{http://dx.doi.org/10.1177/1045389X03038841}{A Free Energy Model for Hysteresis in Ferroelectric Materials}. Journal of Intelligent Material Systems and Structures, Nov 2003;  14:  719 - 739.

Smith, R.C., Salapaka, M.V., Hatch, A., Smith, J.R., De, T. \href{http://dx.doi.org/10.1109/CDC.2002.1184930}{Model Development and Inverse Compensator Design for High Speed Nanopositioning}. Proceedings of the 41st IEEE Conference on Decision and Control, 2002.Volume: 4,10-13 Dec. 2002 Pages:3652  - 3657 vol.4


\section{\sc Presentations}
Smith, J.R., Ricketts, D., Hu, W., Dang, Y., Ozcan, O., Sitti, M., Davis, R., Bain, J. November 2007. Scanning Probe Nanomanufacturing on Si: Surface Characterization of the Process Technique. Materials Research Society Fall Meeting, Boston, Massachusetts. (Abstract accepted, presentation pending)

Smith, J.R., Bilbro, G., Nemanich, R. March 2009. Optimized vacuum thermionic energy conversion using diamond materials. American Physical Society March Meeting, Pittsburgh, Pennsylvania.

Smith, J.R., Bilbro, G., Nemanich, R. November 2007. Vacuum thermionic energy conversion from nitrogen and phosphorus doped diamond. Materials Research Society Fall Meeting, Boston, Massachusetts.

Smith, J.R., Nemanich, R., Friedmann, T., Hertzberg, E. Development of a Hydrogen Termination Procedure for Tetrahedral Amorphous Carbon for use with the Interstellar Boundary Explorer. Materials Research Society Fall Meeting, Boston, Massachusetts.

Smith, J.R., Bilbro, G., Nemanich, R. September 2007. Efficient conversion of heat directly to electricity using negative electron affinity diamond electrodes. 18th European Conference on Diamond, Diamond-Like Materials, Carbon Nanotubes, and Nitrides 2007. Berlin, Germany.

Smith, J.R., Bilbro, G., Nemanich, R. March 2007. Theoretical investigation of vacuum thermionic energy conversion devices for efficient conversion of solar to electrical energy. American Physical Society March Meeting, Denver, Colorado.

Smith, J.R., Bilbro, G., Nemanich, R. May 2006. The Effect of Negative Electron Affinity Emitters on the Space Charge Effect of Vacuum Thermionic Energy Conversion Devices. ICNDST \& ADC 2006 Joint Conference, Research Triangle Park, North Carolina.

Smith, J.R., Bilbro, G., Nemanich, R. March 2006. The Effect of Negative Electron Affinity Emitter Materials on Space Charge Mitigation of Vacuum Thermionic Energy Conversion Devices. American Physical Society March Meeting, Baltimore, Maryland.

Smith, J.R., Bilbro, G., Nemanich, R. December 2005. Vacuum TEC Modeling. Thermionic Energy Conversion MURI Review Meeting. Berkeley, California.

Smith, J.R., Bilbro, G., Nemanich, R. November 2005. Effect of Nanostructured Emitters on the Performance of Vacuum Thermionic Energy Conversion Devices. Materials Research Society Fall Meeting, Boston, Massachusetts.

Smith, J.R., Bilbro, G., Nemanich, R. December 2004. Modeling Vacuum Thermionic Energy Converters. Thermionic Energy Conversion MURI Review Meeting. Santa Cruz, California.

Smith, J.R., Bilbro, G., Nemanich, R. July 2004. Modeling Vacuum Thermionic Energy Converters. Thermionic Energy Conversion MURI Review Meeting, Raleigh, North Carolina.

Smith, J.R., Bilbro, G., Nemanich, R. April 2004. The Theory of Thermionic Energy Conversion. Thermioinic Energy Conversion SBIR phase II Kickoff meeting, Raleigh, North Carolina.

Smith, J.R. January 2004. A Brief Introduction to MATLAB. Graduate Physics Student Association meeting, Raleigh, North Carolina.

Smith, J.R. and Mitas, L. 2001. Molecular Dynamics Simulations. 2001 Summer REU Program Presentations, Raleigh, North Carolina.


\section{\sc Posters}
Smith, J.R., Ricketts, D., Davis, R., Bain, J., Fedder, G., Sitti, M., Santhanam, S., Dang, Y., Hu, W., Ozcan, O., Zhang, A., Gu, J. Tip directed, field assisted nanomanufacturing. DARPA MEMS PI Review Meeting July 2009. San Francisco, California.

Smith, J.R., Hu, W., Dang, Y., Ozcan, O., Sitti, M., Bain, J., Davis, R., Ricketts, D. Towards Writing Si Nanowires on Si (100) with an STM Tip: Surface Preparation and Initial Results. Materials Research Society Fall Meeting 2009. Boston, Massachusetts.

Smith, J.R., Ricketts, D., Davis, R., Bain, J., Fedder, G., Sitti, M., Santhanam, S., Dang, Y., Hu, W., Ozcan, O., Zhang, A. Tip directed, field assisted nanomanufacturing: Initial surface preparation results. DARPA MEMS PI Review Meeting July 2009. Sunriver, Oregon.

Smith, J.R., Nemanich, R. Hertzberg, E., Friedmann, T.A. Hydrogen termination of ta:C for use in interstellar neutral particle detection. New Diamond and Nano Carbons 2007. Osaka, Japan.

Smith, J.R., Nemanich, R. Effect of Hydrogen Passivation on RMS Roughness and Electronic Structure of Diamond-like Carbon Films. Materials Research Society Fall Meeting 2006. Boston, Massachusetts.

Smith, J.R., Bilbro, G., Nemanich, R. Theory of the performance of a thermionic energy conversion device with a negative electron affinity emitter. 17th European Conference on Diamond, Diamond-Like Materials, Carbon Nanotubes, and Nitrides 2006. Estoril, Portugal.

% Smith, J.R., Bilbro, G., Nemanich, R. ONR MURI Dec 2005

Smith, J.R., Bilbro, G., Nemanich, R. A model for the effect of Schottky barrier lowering and non-planar emitter geometry on the performance of a thermionic energy converter. 16th European Conference on Diamond, Diamond-Like Materials, Carbon Nanotubes, and Nitrides 2005. Toulouse, France.

Smith, J.R., Bilbro, G., Nemanich, R. Modeling Thermionic Energy Conversion Devices. June 2005 Thermionic Energy Conversion MURI Meeting, Santa Barbara, California.

Smith, J.R. and Bilbro, G. 2003. Conventional Theory of Thermioinic Emission. 2003 Thermionic Energy Conversion MURI Review Meeting, Cambridge, Massachusetts.


\end{resume}
\end{document}




