% \section{Joshua Ryan Smith}

% \href{mailto:joshua.r.smith@gmail.com}{joshua.r.smith@gmail.com}\\(919)
% 413.0396\\\href{http://orcid.org/0000-0002-3137-7180}{ORCID}\\\href{http://github.com/jrsmith3}{github}

\section{Summary}

I have expertise in developing physics-based models of physical systems
and developing the software to implement them. In the past, I have
worked on nanolithography using a scanning tunneling microscope as a
stylus, and surface science including x-ray photoemission spectroscopy,
Auger electron spectroscopy, and low-energy electron diffraction.

I study energy conversion devices based on wide-bandgap semiconductors,
specifically the III-nitrides. I am particularly interested in
thermoelectron emission and energy conversion (sometimes called
thermionic emission).

\section{Experience}

\textbf{US Army Research Laboratory} - ORAU Senior Researcher, Sensors
and Electron Devices Directorate (August 2011 - Present).

At ARL I have investigated radioisotope batteries, namely betavoltaic
and betaphotovoltaic devices. Our team demonstrated a working GaN based
betavoltaic device -- I specified the structure, verified the devices
using electron beam induced current scanning tunneling microscopy, and
guided the devices through the various fabrication and characterization
steps. For the betaphotovoltaic part of the project, I specified the
AlGaN photovoltaic structure and developed a process of electrophoretic
deposition to apply phosphor to the photovoltaic devices
post-fabrication.

In addition to the radioisotope batteries, I developed a model of
electron transport through a thermoelectron energy conversion device
featuring a negative electron affinity anode. The software I wrote for
this project is used by university research groups and a startup
company. This work was
\href{http://www.arl.army.mil/www/default.cfm?article=2462}{recognized
by ARL} as well as the
\href{http://www.army.mil/article/123473/Visiting_Army_scientist_makes_discoveries_in_emerging_technology/}{official
homepage of the US Army}.

\textbf{Carnegie Mellon University} - Postdoctoral Researcher, Prof.
Robert Davis (June 2008 - July 2011).

At CMU I was the lead postdoc on a project to develop a nanolithography
technique using a scanning tunneling microscope tip as a stylus. During
this time I built the \href{https://www.flickr.com/groups/tfan/}{TFAN
nanolithography/surface science laboratory} from scratch. This lab
featured an ultrahigh vacuum surface analysis system including in-situ
sample preparation, scanning probe microscopy, x-ray photoemission
spectroscopy, Auger electron spectroscopy, and low-energy electron
diffraction. I managed three graduate students and we developed
processes to clean and hydrogen passivate Si (100), write features in
the hydrogen passivation with the scanning probe tip, and adsorb
disilane to the features.

\textbf{NC State University} - Graduate Research Assistant, Profs.
Robert J. Nemanich and Griff L. Bilbro (August 2002 - August 2007).

At NCSU I investigated thermionic energy conversion devices based on
diamond electrodes. I developed a comprehensive model of electron
transport through the device which could account for emission from a
diamond cathode featuring negative electron affinity. I implemented the
model initially in MATLAB and later refactored in python. I also
investigated negative electron affinity amorphous carbon films for the
\href{http://www.nasa.gov/mission_pages/ibex/index.html}{Interstellar
Boundary Explorer (IBEX)} mission in collaboration with Lockheed Martin.
I developed a hydrogen passivation process for the amorphous carbon
films and characterized the negative electron affinity with ultraviolet
photoemission spectroscopy. I applied this process to amorphous carbon
facets which were used as part of the IBEX-Lo detector. The IBEX space
probe was launched October 19, 2008.

\textbf{NC State University} - Undergraduate Researcher, Prof. Lubos
Mitas (August 2001 - May 2002).

Monte Carlo computational molecular dynamics.

\textbf{NC State University} - Research Experience for Undergraduates
Program, Prof. Lubos Mitas (May 2001 - August 2001).

Monte Carlo computational molecular dynamics

\textbf{Physics Tutorial Center, NC State University} - Tutor (August
1998 -May 2000).

\textbf{Triangle Learning Consultants,} Raleigh, NC - Tutor (August 2000
-August 2002).

High school mathematics tutoring.

\textbf{Johnson Controls,} Charlotte, NC - Engineering Assistant (Summer
1999, Summer 2000).

AutoCAD

\section{Education}

\begin{itemize}
\item
  \textbf{Ph.D.} Physics, NC State University, 2007
\item
  \textbf{B.S.} Physics, Cum Laude, NC State University, 2002
\item
  \textbf{B.S.} Mathematics, Cum Laude, NC State University, 2002
\item
  \textbf{High School} North Carolina School of Science and Mathematics,
  1998
\end{itemize}

\section{Skills and capabilities}

My specialty is developing mathematical models of physical systems and
implementing them in software using industry-standard software
development practices. I have experience with the version control
systems subversion and git, the programming languages python and MATLAB,
SQLite databases, continuous integration with Travis-CI, and project
management with GitHub. I also have extensive experience with
experimental surface science, ultrahigh vacuum equipment, and laboratory
operations.

\section{Teaching}

\begin{itemize}
\item
  \href{https://pages.nist.gov/2015-09-23-nist/}{Software Carpentry
  Workshop, National Institute of Standards and Technology, Gaithersburg
  MD. September 23-24, 2015}
\item
  \href{https://pages.nist.gov/2015-07-23-nist/}{Software Carpentry
  Workshop, National Institute of Standards and Technology, Gaithersburg
  MD. July 23-24, 2015}
\item
  Software Carpentry Bootcamp, Carnegie Mellon University. July 27-28,
  2013
\item
  Software Carpentry Bootcamp, Johns Hopkins University. June 18-19,
  2012
\item
  Software Carpentry Bootcamp, University of Chicago. April 2-3, 2012
\end{itemize}

\section{Honors and Awards}

\begin{itemize}
\item
  COMAP Mathematical Contest in Modeling 2002, Meritorious Submission
\item
  Eagle Scout Award, 1996
\end{itemize}

\section{Professional Affiliations}

\begin{itemize}
\item
  Member, Materials Research Society
\item
  Member, American Physical Society
\end{itemize}

\section{Service}

\begin{itemize}
\item
  President, Graduate Physics Student Association (GPSA). April 2005
  -April 2006
\item
  University Graduate Student Assc. Representative. April 2003 - April
  2005
\end{itemize}

\section{Countries Visited}

Canada, France, Georgia, Germany, Greece, Italy, Morocco, Netherlands,
Poland, Portugal, Spain, Turkey, Ukraine.

\section{Dissertation}

\href{http://www.lib.ncsu.edu/resolver/1840.16/3107}{Thermionic Energy
Conversion and Particle Detection Using Diamond and Diamond-Like Carbon
Surfaces}

Committee: Robert J. Nemanich (co-chair), Griff Bilbro (co-chair), David
Aspnes, Thomas Perl

\section{Selected Software}

\begin{itemize}
\item
  \href{http://jrsmith3.github.io/tec/}{tec} - Utilities for simulating
  vacuum thermionic energy conversion devices.
\item
  \href{http://ibei.readthedocs.org/en/latest/}{ibei} - Calculator for
  incomplete Bose-Einstein integral.
\end{itemize}

\section{Publications}

Tompkins, R.P., \textbf{Smith, J.R.}, Kirchner, K.W., Jones, K.A.,
Leach, J.H., Udwary, K., Preble, E., Suvarna, P., Leathersich, J.M.,
Shahedipour-Sandvik, F.
\href{http://dx.doi.org/10.1007/s11664-014-3021-9}{GaN Power Schottky
Diodes with Drift Layers Grown on Four Substrates}. Journal of
Electronic Materials, 2014; 43 (4): 850-856

\textbf{Smith, J.R.}
\href{http://dx.doi.org/10.1063/1.4826202}{Increasing the efficiency of
a thermionic engine using a negative electron affinity collector}.
Journal of Applied Physics, 2013; 114: 164514

Tompkins, R.P., \textbf{Smith, J.R.}, Kirchner, K.W., Jones, K.A.,
Preble, E., Leach, J., Mulholland, G., Suvarna, P., Tungare, M.,
Shahedipour-Sandvik, F. \href{http://dx.doi.org/10.1149/1.3701521}{GaN
Power Schottky Diodes}. ECS Transactions 2012; 45 (7): 17-25

\textbf{Smith, J.R.}, Bilbro, G., Nemanich, R.
\href{http://dx.doi.org/10.1116/1.3125282}{Theory of space charge
limited regime of thermionic energy converter with negative electron
affinity emitter}. Journal of Vacuum Science and Technology B, 2009; 27:
1132-1141

\textbf{Smith, J.R.}, Bilbro, G., Nemanich, R.
\href{http://dx.doi.org/10.1103/PhysRevB.76.245327}{Considerations for a
high performance thermionic energy conversion device based on an NEA
emitter}. Physical Review B, 2007; 76: 245327-245332

\textbf{Smith, J.R.}, Bilbro, G., Nemanich, R.
\href{http://dx.doi.org/10.1016/j.diamond.2006.09.011}{Using negative
electron affinity diamond emitters to mitigate space charge in vacuum
thermionic energy conversion devices}. Diamond and Related Materials,
2006; 15: 2082-2085.

\textbf{Smith, J.R.}, Bilbro, G., Nemanich, R.
\href{http://dx.doi.org/10.1016/j.diamond.2005.12.057}{The effect of
Schottky barrier lowering and nonplanar emitter geometry on the
performance of a thermionic energy converter}. Diamond and Related
Materials, 2006; 15: 870-874.

Smith, R.C., Seelecke, S., Ounaies, Z., \textbf{Smith, J.R.}
\href{http://dx.doi.org/10.1177/1045389X03038841}{A Free Energy Model
for Hysteresis in Ferroelectric Materials}. Journal of Intelligent
Material Systems and Structures, Nov 2003; 14: 719 - 739.

Smith, R.C., Salapaka, M.V., Hatch, A., \textbf{Smith, J.R.}, De, T.
\href{http://dx.doi.org/10.1109/CDC.2002.1184930}{Model Development and
Inverse Compensator Design for High Speed Nanopositioning}. Proceedings
of the 41st IEEE Conference on Decision and Control, 2002.Volume:
4,10-13 Dec. 2002 Pages:3652 - 3657 vol.4

\section{Invited Presentations}

\textbf{Smith, J.R.} September 2015. Achieving \textgreater{}20\%
efficiency using a vacuum thermionic energy converter featuring a
diamond anode. University of British Columbia.

\section{Presentations}

\textbf{Smith, J.R.} November 2013. Achieving \textgreater{}20\%
efficiency using a vacuum thermionic energy converter featuring a
III-nitride, negative electron affinity anode. Materials Research
Society Fall Meeting, Boston, Massachusetts.

\textbf{Smith, J.R.}, Ricketts, D., Bain, J. June 2011. Localized
Thermal Modification of Surfaces via Electron Bombardment from an STM
Tip. 55th International Conference on Electron, Ion, and Photon Beam
Technology and Nanofabrication, Las Vegas, Nevada.

\textbf{Smith, J.R.}, Ricketts, D., Hu, W., Dang, Y., Ozcan, O., Sitti,
M., Davis, R., Bain, J. November 2010. Scanning Probe Nanomanufacturing
on Si: Surface Characterization of the Process Technique. Materials
Research Society Fall Meeting, Boston, Massachusetts.

\textbf{Smith, J.R.}, Bilbro, G., Nemanich, R. March 2009. Optimized
vacuum thermionic energy conversion using diamond materials. American
Physical Society March Meeting, Pittsburgh, Pennsylvania.

\textbf{Smith, J.R.}, Bilbro, G., Nemanich, R. November 2007. Vacuum
thermionic energy conversion from nitrogen and phosphorus doped diamond.
Materials Research Society Fall Meeting, Boston, Massachusetts.

\textbf{Smith, J.R.}, Nemanich, R., Friedmann, T., Hertzberg, E.
November 2007. Development of a Hydrogen Termination Procedure for
Tetrahedral Amorphous Carbon for use with the Interstellar Boundary
Explorer. Materials Research Society Fall Meeting, Boston,
Massachusetts.

\textbf{Smith, J.R.}, Bilbro, G., Nemanich, R. September 2007. Efficient
conversion of heat directly to electricity using negative electron
affinity diamond electrodes. 18th European Conference on Diamond,
Diamond-Like Materials, Carbon Nanotubes, and Nitrides 2007. Berlin,
Germany.

\textbf{Smith, J.R.}, Bilbro, G., Nemanich, R. March 2007. Theoretical
investigation of vacuum thermionic energy conversion devices for
efficient conversion of solar to electrical energy. American Physical
Society March Meeting, Denver, Colorado.

\textbf{Smith, J.R.}, Bilbro, G., Nemanich, R. May 2006. The Effect of
Negative Electron Affinity Emitters on the Space Charge Effect of Vacuum
Thermionic Energy Conversion Devices. ICNDST \& ADC 2006 Joint
Conference, Research Triangle Park, North Carolina.

\textbf{Smith, J.R.}, Bilbro, G., Nemanich, R. March 2006. The Effect of
Negative Electron Affinity Emitter Materials on Space Charge Mitigation
of Vacuum Thermionic Energy Conversion Devices. American Physical
Society March Meeting, Baltimore, Maryland.

\textbf{Smith, J.R.}, Bilbro, G., Nemanich, R. December 2005. Vacuum TEC
Modeling. Thermionic Energy Conversion MURI Review Meeting. Berkeley,
California.

\textbf{Smith, J.R.}, Bilbro, G., Nemanich, R. November 2005. Effect of
Nanostructured Emitters on the Performance of Vacuum Thermionic Energy
Conversion Devices. Materials Research Society Fall Meeting, Boston,
Massachusetts.

\textbf{Smith, J.R.}, Bilbro, G., Nemanich, R. December 2004. Modeling
Vacuum Thermionic Energy Converters. Thermionic Energy Conversion MURI
Review Meeting. Santa Cruz, California.

\textbf{Smith, J.R.}, Bilbro, G., Nemanich, R. July 2004. Modeling
Vacuum Thermionic Energy Converters. Thermionic Energy Conversion MURI
Review Meeting, Raleigh, North Carolina.

\textbf{Smith, J.R.}, Bilbro, G., Nemanich, R. April 2004. The Theory of
Thermionic Energy Conversion. Thermioinic Energy Conversion SBIR phase
II Kickoff meeting, Raleigh, North Carolina.

\textbf{Smith, J.R.} and Mitas, L. 2001. Molecular Dynamics Simulations.
2001 Summer REU Program Presentations, Raleigh, North Carolina.

\section{Posters}

\textbf{Smith, J.R.} November 2014.
\href{https://github.com/jrsmith3/conf-mrs_fall_2014_poster/releases}{Beta-enhanced
thermoelectron emission and energy conversion}, Boston, MA.

\textbf{Smith, J.R.} August 2013. Simulated thermionic engine
performance using III-nitride, negative electron affinity collector,
Washington, DC.

\textbf{Smith, J.R.}, Ricketts, D., Davis, R., Bain, J., Fedder, G.,
Sitti, M., Santhanam, S., Dang, Y., Hu, W., Ozcan, O., Zhang, A., Gu, J.
Tip directed, field assisted nanomanufacturing. DARPA MEMS PI Review
Meeting July 2010. San Francisco, California.

\textbf{Smith, J.R.}, Hu, W., Dang, Y., Ozcan, O., Sitti, M., Bain, J.,
Davis, R., Ricketts, D. Towards Writing Si Nanowires on Si (100) with an
STM Tip: Surface Preparation and Initial Results. Materials Research
Society Fall Meeting 2009. Boston, Massachusetts.

\textbf{Smith, J.R.}, Ricketts, D., Davis, R., Bain, J., Fedder, G.,
Sitti, M., Santhanam, S., Dang, Y., Hu, W., Ozcan, O., Zhang, A. Tip
directed, field assisted nanomanufacturing: Initial surface preparation
results. DARPA MEMS PI Review Meeting July 2009. Sunriver, Oregon.

\textbf{Smith, J.R.}, Nemanich, R. Hertzberg, E., Friedmann, T.A.
Hydrogen termination of ta:C for use in interstellar neutral particle
detection. New Diamond and Nano Carbons 2007. Osaka, Japan.

\textbf{Smith, J.R.}, Nemanich, R. Effect of Hydrogen Passivation on RMS
Roughness and Electronic Structure of Diamond-like Carbon Films.
Materials Research Society Fall Meeting 2006. Boston, Massachusetts.

\textbf{Smith, J.R.}, Bilbro, G., Nemanich, R. Theory of the performance
of a thermionic energy conversion device with a negative electron
affinity emitter. 17th European Conference on Diamond, Diamond-Like
Materials, Carbon Nanotubes, and Nitrides 2006. Estoril, Portugal.

\textbf{Smith, J.R.}, Bilbro, G., Nemanich, R. A model for the effect of
Schottky barrier lowering and non-planar emitter geometry on the
performance of a thermionic energy converter. 16th European Conference
on Diamond, Diamond-Like Materials, Carbon Nanotubes, and Nitrides 2005.
Toulouse, France.

\textbf{Smith, J.R.}, Bilbro, G., Nemanich, R. Modeling Thermionic
Energy Conversion Devices. June 2005 Thermionic Energy Conversion MURI
Meeting, Santa Barbara, California.

\textbf{Smith, J.R.} and Bilbro, G. Conventional Theory of Thermioinic
Emission. November 2003. Thermionic Energy Conversion MURI Review
Meeting, Cambridge, Massachusetts.
